\documentclass[]{article}
\usepackage{lmodern}
\usepackage{amssymb,amsmath}
\usepackage{ifxetex,ifluatex}
\usepackage{fixltx2e} % provides \textsubscript
\ifnum 0\ifxetex 1\fi\ifluatex 1\fi=0 % if pdftex
  \usepackage[T1]{fontenc}
  \usepackage[utf8]{inputenc}
\else % if luatex or xelatex
  \ifxetex
    \usepackage{mathspec}
  \else
    \usepackage{fontspec}
  \fi
  \defaultfontfeatures{Ligatures=TeX,Scale=MatchLowercase}
\fi
% use upquote if available, for straight quotes in verbatim environments
\IfFileExists{upquote.sty}{\usepackage{upquote}}{}
% use microtype if available
\IfFileExists{microtype.sty}{%
\usepackage{microtype}
\UseMicrotypeSet[protrusion]{basicmath} % disable protrusion for tt fonts
}{}
\usepackage[margin=1in]{geometry}
\usepackage{hyperref}
\hypersetup{unicode=true,
            pdftitle={Regresion},
            pdfborder={0 0 0},
            breaklinks=true}
\urlstyle{same}  % don't use monospace font for urls
\usepackage{color}
\usepackage{fancyvrb}
\newcommand{\VerbBar}{|}
\newcommand{\VERB}{\Verb[commandchars=\\\{\}]}
\DefineVerbatimEnvironment{Highlighting}{Verbatim}{commandchars=\\\{\}}
% Add ',fontsize=\small' for more characters per line
\usepackage{framed}
\definecolor{shadecolor}{RGB}{248,248,248}
\newenvironment{Shaded}{\begin{snugshade}}{\end{snugshade}}
\newcommand{\AlertTok}[1]{\textcolor[rgb]{0.94,0.16,0.16}{#1}}
\newcommand{\AnnotationTok}[1]{\textcolor[rgb]{0.56,0.35,0.01}{\textbf{\textit{#1}}}}
\newcommand{\AttributeTok}[1]{\textcolor[rgb]{0.77,0.63,0.00}{#1}}
\newcommand{\BaseNTok}[1]{\textcolor[rgb]{0.00,0.00,0.81}{#1}}
\newcommand{\BuiltInTok}[1]{#1}
\newcommand{\CharTok}[1]{\textcolor[rgb]{0.31,0.60,0.02}{#1}}
\newcommand{\CommentTok}[1]{\textcolor[rgb]{0.56,0.35,0.01}{\textit{#1}}}
\newcommand{\CommentVarTok}[1]{\textcolor[rgb]{0.56,0.35,0.01}{\textbf{\textit{#1}}}}
\newcommand{\ConstantTok}[1]{\textcolor[rgb]{0.00,0.00,0.00}{#1}}
\newcommand{\ControlFlowTok}[1]{\textcolor[rgb]{0.13,0.29,0.53}{\textbf{#1}}}
\newcommand{\DataTypeTok}[1]{\textcolor[rgb]{0.13,0.29,0.53}{#1}}
\newcommand{\DecValTok}[1]{\textcolor[rgb]{0.00,0.00,0.81}{#1}}
\newcommand{\DocumentationTok}[1]{\textcolor[rgb]{0.56,0.35,0.01}{\textbf{\textit{#1}}}}
\newcommand{\ErrorTok}[1]{\textcolor[rgb]{0.64,0.00,0.00}{\textbf{#1}}}
\newcommand{\ExtensionTok}[1]{#1}
\newcommand{\FloatTok}[1]{\textcolor[rgb]{0.00,0.00,0.81}{#1}}
\newcommand{\FunctionTok}[1]{\textcolor[rgb]{0.00,0.00,0.00}{#1}}
\newcommand{\ImportTok}[1]{#1}
\newcommand{\InformationTok}[1]{\textcolor[rgb]{0.56,0.35,0.01}{\textbf{\textit{#1}}}}
\newcommand{\KeywordTok}[1]{\textcolor[rgb]{0.13,0.29,0.53}{\textbf{#1}}}
\newcommand{\NormalTok}[1]{#1}
\newcommand{\OperatorTok}[1]{\textcolor[rgb]{0.81,0.36,0.00}{\textbf{#1}}}
\newcommand{\OtherTok}[1]{\textcolor[rgb]{0.56,0.35,0.01}{#1}}
\newcommand{\PreprocessorTok}[1]{\textcolor[rgb]{0.56,0.35,0.01}{\textit{#1}}}
\newcommand{\RegionMarkerTok}[1]{#1}
\newcommand{\SpecialCharTok}[1]{\textcolor[rgb]{0.00,0.00,0.00}{#1}}
\newcommand{\SpecialStringTok}[1]{\textcolor[rgb]{0.31,0.60,0.02}{#1}}
\newcommand{\StringTok}[1]{\textcolor[rgb]{0.31,0.60,0.02}{#1}}
\newcommand{\VariableTok}[1]{\textcolor[rgb]{0.00,0.00,0.00}{#1}}
\newcommand{\VerbatimStringTok}[1]{\textcolor[rgb]{0.31,0.60,0.02}{#1}}
\newcommand{\WarningTok}[1]{\textcolor[rgb]{0.56,0.35,0.01}{\textbf{\textit{#1}}}}
\usepackage{graphicx,grffile}
\makeatletter
\def\maxwidth{\ifdim\Gin@nat@width>\linewidth\linewidth\else\Gin@nat@width\fi}
\def\maxheight{\ifdim\Gin@nat@height>\textheight\textheight\else\Gin@nat@height\fi}
\makeatother
% Scale images if necessary, so that they will not overflow the page
% margins by default, and it is still possible to overwrite the defaults
% using explicit options in \includegraphics[width, height, ...]{}
\setkeys{Gin}{width=\maxwidth,height=\maxheight,keepaspectratio}
\IfFileExists{parskip.sty}{%
\usepackage{parskip}
}{% else
\setlength{\parindent}{0pt}
\setlength{\parskip}{6pt plus 2pt minus 1pt}
}
\setlength{\emergencystretch}{3em}  % prevent overfull lines
\providecommand{\tightlist}{%
  \setlength{\itemsep}{0pt}\setlength{\parskip}{0pt}}
\setcounter{secnumdepth}{0}
% Redefines (sub)paragraphs to behave more like sections
\ifx\paragraph\undefined\else
\let\oldparagraph\paragraph
\renewcommand{\paragraph}[1]{\oldparagraph{#1}\mbox{}}
\fi
\ifx\subparagraph\undefined\else
\let\oldsubparagraph\subparagraph
\renewcommand{\subparagraph}[1]{\oldsubparagraph{#1}\mbox{}}
\fi

%%% Use protect on footnotes to avoid problems with footnotes in titles
\let\rmarkdownfootnote\footnote%
\def\footnote{\protect\rmarkdownfootnote}

%%% Change title format to be more compact
\usepackage{titling}

% Create subtitle command for use in maketitle
\providecommand{\subtitle}[1]{
  \posttitle{
    \begin{center}\large#1\end{center}
    }
}

\setlength{\droptitle}{-2em}

  \title{Regresion}
    \pretitle{\vspace{\droptitle}\centering\huge}
  \posttitle{\par}
    \author{}
    \preauthor{}\postauthor{}
    \date{}
    \predate{}\postdate{}
  

\begin{document}
\maketitle

\hypertarget{tarea-3.}{%
\section{Tarea 3.}\label{tarea-3.}}

\hypertarget{regresion-lineal}{%
\section{Regresión lineal}\label{regresion-lineal}}

Análisis del Problema

El desempeño de un automóvil se puede medir de diferentes formas.
Algunas comunes son la cantidad de \textbf{caballos de fuerza} y el
\textbf{rendimiento} del mismo, que se puede resumir en cuantas millas
puede recorrer el automóvil por cada galón de combustible que consume.
Para los clientes, potenciales compradores de un automóvil, este
rendimiento es importante pues puede ayudar a tomar una decisión con
respecto a cuál automóvil comprar (si, por ejemplo, el cliente quiere un
auto que rinda por muchas millas y pueda economizar en la compra de
combustible).

Desde este punto de vista, tanto a clientes como a fabricadores de
automóviles, les conviene entender cuál es la relación entre diferentes
características del automóvil y su rendimiento, pues el conocer estas
relaciones les puede ayudar a inferir cuál va a ser la eficiencia del
vehículo a partir de ver los valores de otras características. Para
fabricantes, puede ser importante conocer estas relaciones para saber
cómo hacer cada modelo más eficiente con respecto al anterior.

Entendimiento de los Datos

Con el fin de analizar y tratar de estimar las millas por galón de
diferentes modelos de automóviles, se trabajó con un conjunto de datos
que contiene 398 observaciones y 9 variables:

\begin{itemize}
\tightlist
\item
  mpg (millas por galón): numérica, con un rango de 9 a 46.60.
\item
  cyl (cilindraje): categórica ordinal, con valores posibles de 3, 4, 5,
  6 y 8.
\item
  disp (desplazamiento): numérica, con un rango de 68 a 455.
\item
  hp (caballos de fuerza): numérica, con un rango de 46 a 230 y 6
  valores faltantes.
\item
  weight (peso): numérica, con un rango de 1613 a 5140.
\item
  acc (aceleración): numérica, con un rango de 8 a 24.80.
\item
  model year (año): categórica, con 13 valores diferentes representando
  el año del automóvil.
\item
  origin (origen): categórica, 3 valores posibles: 1, 2, 3.
\item
  model name (nombre del modelo): categórica, con 305 posibles valores.
\end{itemize}

\hypertarget{ejercicios}{%
\section{Ejercicios}\label{ejercicios}}

\begin{verbatim}
## Loading required package: ggplot2
\end{verbatim}

\begin{verbatim}
## Registered S3 method overwritten by 'GGally':
##   method from   
##   +.gg   ggplot2
\end{verbatim}

\begin{verbatim}
## 
## lessR 3.8.8     feedback: gerbing@pdx.edu     web: lessRstats.com/new
## ---------------------------------------------------------------------
## 1. d <- Read("")           Read text, Excel, SPSS, SAS or R data file
##                            d: default data frame, no need for data=
## 2. l <- Read("", var_labels=TRUE)   Read variable labels into l,
##                            required name for data frame of labels
## 3. Help()                  Get help, and, e.g., Help(Read)
## 4. hs(), bc(), or ca()     All histograms, all bar charts, or both
## 5. Plot(X) or Plot(X,Y)    For continuous and categorical variables
## 6. by1= , by2=             Trellis graphics, a plot for each by1, by2
## 7. reg(Y ~ X, Rmd="eg")    Regression with full interpretative output
## 8. style("gray")           Grayscale theme, + many others available
##    style(show=TRUE)        all color/style options and current values
## 9. getColors()             create many styles of color palettes
## 
## lessR parameter names now use _'s. Names with a period are deprecated.
## Ex:  bin_width  instead of  bin.width
\end{verbatim}

\begin{verbatim}
## 
## Attaching package: 'mltools'
\end{verbatim}

\begin{verbatim}
## The following objects are masked from 'package:Metrics':
## 
##     mse, msle, rmse, rmsle
\end{verbatim}

\begin{verbatim}
## 
## Attaching package: 'dplyr'
\end{verbatim}

\begin{verbatim}
## The following object is masked from 'package:GGally':
## 
##     nasa
\end{verbatim}

\begin{verbatim}
## The following objects are masked from 'package:stats':
## 
##     filter, lag
\end{verbatim}

\begin{verbatim}
## The following objects are masked from 'package:base':
## 
##     intersect, setdiff, setequal, union
\end{verbatim}

\begin{verbatim}
## 
## Attaching package: 'tidyr'
\end{verbatim}

\begin{verbatim}
## The following object is masked from 'package:mltools':
## 
##     replace_na
\end{verbatim}

\begin{enumerate}
\def\labelenumi{\arabic{enumi}.}
\tightlist
\item
  Cargue el archivo auto-mpg\_g.csv en una variable
\end{enumerate}

\begin{Shaded}
\begin{Highlighting}[]
\NormalTok{autos <-}\StringTok{ }\KeywordTok{read_csv}\NormalTok{(}\StringTok{"auto-mpg_g.csv"}\NormalTok{)}
\end{Highlighting}
\end{Shaded}

\begin{verbatim}
## Parsed with column specification:
## cols(
##   mpg = col_double(),
##   cyl = col_double(),
##   disp = col_double(),
##   hp = col_double(),
##   weight = col_double(),
##   acc = col_double(),
##   model.year = col_double(),
##   origin = col_double(),
##   model.name = col_character()
## )
\end{verbatim}

\begin{Shaded}
\begin{Highlighting}[]
\KeywordTok{head}\NormalTok{(autos, }\DecValTok{400}\NormalTok{)}
\end{Highlighting}
\end{Shaded}

\begin{verbatim}
## # A tibble: 398 x 9
##      mpg   cyl  disp    hp weight   acc model.year origin model.name       
##    <dbl> <dbl> <dbl> <dbl>  <dbl> <dbl>      <dbl>  <dbl> <chr>            
##  1    18     8   307   130   3504    12         70      1 chevrolet chevel~
##  2    15     8   350   165   3693    12         70      1 buick skylark 320
##  3    18     8   318   150   3436    11         70      1 plymouth satelli~
##  4    16     8   304   150   3433    12         70      1 amc rebel sst    
##  5    17     8   302   140   3449    11         70      1 ford torino      
##  6    15     8   429   198   4341    10         70      1 ford galaxie 500 
##  7    14     8   454   220   4354     9         70      1 chevrolet impala 
##  8    14     8   440   215   4312     9         70      1 plymouth fury iii
##  9    14     8   455   225   4425    10         70      1 pontiac catalina 
## 10    15     8   390   190   3850     9         70      1 amc ambassador d~
## # ... with 388 more rows
\end{verbatim}

\begin{enumerate}
\def\labelenumi{\arabic{enumi}.}
\setcounter{enumi}{1}
\tightlist
\item
  Utilizando Ggpairs cree un gráfico de los atributos del dataset,
  observe las correlaciones entre atributos
\end{enumerate}

\textbf{Preparacion de los datos:}

\begin{Shaded}
\begin{Highlighting}[]
\NormalTok{get.mode <-}\StringTok{ }\ControlFlowTok{function}\NormalTok{(v) \{}
\NormalTok{   uniqv <-}\StringTok{ }\KeywordTok{unique}\NormalTok{(v)}
\NormalTok{   uniqv[}\KeywordTok{which.max}\NormalTok{(}\KeywordTok{tabulate}\NormalTok{(}\KeywordTok{match}\NormalTok{(v, uniqv)))]}
\NormalTok{\}}
\CommentTok{# Reemplazar los valores en "0" dentro de la columna caballos de fuerza.}
\NormalTok{hp_mode <-}\StringTok{ }\KeywordTok{get.mode}\NormalTok{(autos}\OperatorTok{$}\NormalTok{hp)}
\NormalTok{autos}\OperatorTok{$}\NormalTok{hp[autos}\OperatorTok{$}\NormalTok{hp }\OperatorTok{==}\StringTok{ }\DecValTok{0}\NormalTok{] <-}\StringTok{ }\NormalTok{hp_mode}

\CommentTok{# Llenar los valores nulos en la columna caballos de fuerza}
\NormalTok{media_hp =}\StringTok{ }\KeywordTok{median}\NormalTok{(autos}\OperatorTok{$}\NormalTok{hp)}
\NormalTok{autos}\OperatorTok{$}\NormalTok{hp[}\KeywordTok{is.na}\NormalTok{(autos}\OperatorTok{$}\NormalTok{hp)] <-}\StringTok{ }\NormalTok{media_hp}

\CommentTok{# Dividir los valores de la columna modelo para extraer solo el fabricante, ya que al tener 305 modelos estabamos perdiendo variabilidad en los datos}
\NormalTok{autos <-}\StringTok{ }\NormalTok{autos }\OperatorTok\StringTok{ }\KeywordTok{separate}\NormalTok{(model.name, }
                \KeywordTok{c}\NormalTok{(}\StringTok{"manufacturer"}\NormalTok{))}
\KeywordTok{head}\NormalTok{(autos, }\DecValTok{400}\NormalTok{)}
\end{Highlighting}
\end{Shaded}

\begin{verbatim}
## # A tibble: 398 x 9
##      mpg   cyl  disp    hp weight   acc model.year origin manufacturer
##    <dbl> <dbl> <dbl> <dbl>  <dbl> <dbl>      <dbl>  <dbl> <chr>       
##  1    18     8   307   130   3504    12         70      1 chevrolet   
##  2    15     8   350   165   3693    12         70      1 buick       
##  3    18     8   318   150   3436    11         70      1 plymouth    
##  4    16     8   304   150   3433    12         70      1 amc         
##  5    17     8   302   140   3449    11         70      1 ford        
##  6    15     8   429   198   4341    10         70      1 ford        
##  7    14     8   454   220   4354     9         70      1 chevrolet   
##  8    14     8   440   215   4312     9         70      1 plymouth    
##  9    14     8   455   225   4425    10         70      1 pontiac     
## 10    15     8   390   190   3850     9         70      1 amc         
## # ... with 388 more rows
\end{verbatim}

\begin{Shaded}
\begin{Highlighting}[]
\KeywordTok{ggpairs}\NormalTok{(autos,}\DataTypeTok{columns =} \KeywordTok{c}\NormalTok{(}\DecValTok{1}\NormalTok{, }\DecValTok{3}\NormalTok{, }\DecValTok{4}\NormalTok{, }\DecValTok{5}\NormalTok{, }\DecValTok{6}\NormalTok{, }\DecValTok{7}\NormalTok{))}
\end{Highlighting}
\end{Shaded}

\includegraphics{Regresion_files/figure-latex/unnamed-chunk-4-1.pdf}

\begin{Shaded}
\begin{Highlighting}[]
\KeywordTok{hist}\NormalTok{(autos}\OperatorTok{$}\NormalTok{mpg)}
\end{Highlighting}
\end{Shaded}

\includegraphics{Regresion_files/figure-latex/unnamed-chunk-5-1.pdf}

\begin{Shaded}
\begin{Highlighting}[]
\KeywordTok{hist}\NormalTok{(autos}\OperatorTok{$}\NormalTok{hp)}
\end{Highlighting}
\end{Shaded}

\includegraphics{Regresion_files/figure-latex/unnamed-chunk-5-2.pdf}

\begin{Shaded}
\begin{Highlighting}[]
\KeywordTok{hist}\NormalTok{(autos}\OperatorTok{$}\NormalTok{disp)}
\end{Highlighting}
\end{Shaded}

\includegraphics{Regresion_files/figure-latex/unnamed-chunk-5-3.pdf}

\begin{Shaded}
\begin{Highlighting}[]
\KeywordTok{hist}\NormalTok{(}\KeywordTok{log10}\NormalTok{(autos}\OperatorTok{$}\NormalTok{mpg), }\DataTypeTok{col=}\StringTok{"#cfbae1"}\NormalTok{,}\DataTypeTok{border=}\StringTok{"#c59fc9"}\NormalTok{)}
\end{Highlighting}
\end{Shaded}

\includegraphics{Regresion_files/figure-latex/unnamed-chunk-5-4.pdf}

\begin{Shaded}
\begin{Highlighting}[]
\KeywordTok{hist}\NormalTok{(}\KeywordTok{log10}\NormalTok{(autos}\OperatorTok{$}\NormalTok{hp), }\DataTypeTok{col=}\StringTok{"#cfbae1"}\NormalTok{,}\DataTypeTok{border=}\StringTok{"#c59fc9"}\NormalTok{)}
\end{Highlighting}
\end{Shaded}

\includegraphics{Regresion_files/figure-latex/unnamed-chunk-5-5.pdf}

\begin{Shaded}
\begin{Highlighting}[]
\KeywordTok{hist}\NormalTok{(}\KeywordTok{log10}\NormalTok{(autos}\OperatorTok{$}\NormalTok{disp), }\DataTypeTok{col=}\StringTok{"#cfbae1"}\NormalTok{,}\DataTypeTok{border=}\StringTok{"#c59fc9"}\NormalTok{)}
\end{Highlighting}
\end{Shaded}

\includegraphics{Regresion_files/figure-latex/unnamed-chunk-5-6.pdf}

\begin{enumerate}
\def\labelenumi{\arabic{enumi}.}
\setcounter{enumi}{2}
\tightlist
\item
  Separe los datos en 2 conjuntos, uno de entrenamiento y otro de
  pruebas. Normalmente se trabaja utilizando un 70-80\% de los datos
  para entrenamiento y el resto para pruebas.
\end{enumerate}

Recuerde fijar una semilla para que el documento sea reproducible.

Pista:
\url{https://www.rdocumentation.org/packages/caTools/versions/1.17.1/topics/sample.split}

\begin{Shaded}
\begin{Highlighting}[]
\KeywordTok{set.seed}\NormalTok{(}\DecValTok{124}\NormalTok{) }\CommentTok{# reproducibilidad}

\NormalTok{y =}\StringTok{  }\NormalTok{dplyr}\OperatorTok{::}\KeywordTok{pull}\NormalTok{(autos, mpg)}

\NormalTok{autos}\OperatorTok{$}\NormalTok{dividir_datos <-}\StringTok{ }\KeywordTok{sample.split}\NormalTok{(y, }\DataTypeTok{SplitRatio=}\FloatTok{0.70}\NormalTok{)}
\KeywordTok{head}\NormalTok{(autos, }\DecValTok{400}\NormalTok{)}
\end{Highlighting}
\end{Shaded}

\begin{verbatim}
## # A tibble: 398 x 10
##      mpg   cyl  disp    hp weight   acc model.year origin manufacturer
##    <dbl> <dbl> <dbl> <dbl>  <dbl> <dbl>      <dbl>  <dbl> <chr>       
##  1    18     8   307   130   3504    12         70      1 chevrolet   
##  2    15     8   350   165   3693    12         70      1 buick       
##  3    18     8   318   150   3436    11         70      1 plymouth    
##  4    16     8   304   150   3433    12         70      1 amc         
##  5    17     8   302   140   3449    11         70      1 ford        
##  6    15     8   429   198   4341    10         70      1 ford        
##  7    14     8   454   220   4354     9         70      1 chevrolet   
##  8    14     8   440   215   4312     9         70      1 plymouth    
##  9    14     8   455   225   4425    10         70      1 pontiac     
## 10    15     8   390   190   3850     9         70      1 amc         
## # ... with 388 more rows, and 1 more variable: dividir_datos <lgl>
\end{verbatim}

\begin{Shaded}
\begin{Highlighting}[]
\NormalTok{autos_training=}\KeywordTok{subset}\NormalTok{(autos, autos}\OperatorTok{$}\NormalTok{dividir_datos}\OperatorTok{==}\OtherTok{TRUE}\NormalTok{)}
\KeywordTok{head}\NormalTok{(autos_training, }\DecValTok{400}\NormalTok{)}
\end{Highlighting}
\end{Shaded}

\begin{verbatim}
## # A tibble: 275 x 10
##      mpg   cyl  disp    hp weight   acc model.year origin manufacturer
##    <dbl> <dbl> <dbl> <dbl>  <dbl> <dbl>      <dbl>  <dbl> <chr>       
##  1    18     8   307   130   3504    12         70      1 chevrolet   
##  2    18     8   318   150   3436    11         70      1 plymouth    
##  3    16     8   304   150   3433    12         70      1 amc         
##  4    17     8   302   140   3449    11         70      1 ford        
##  5    15     8   429   198   4341    10         70      1 ford        
##  6    14     8   454   220   4354     9         70      1 chevrolet   
##  7    14     8   440   215   4312     9         70      1 plymouth    
##  8    15     8   390   190   3850     9         70      1 amc         
##  9    15     8   383   170   3563    10         70      1 dodge       
## 10    14     8   340   160   3609     8         70      1 plymouth    
## # ... with 265 more rows, and 1 more variable: dividir_datos <lgl>
\end{verbatim}

\begin{Shaded}
\begin{Highlighting}[]
\NormalTok{autos_testing=}\KeywordTok{subset}\NormalTok{(autos, autos}\OperatorTok{$}\NormalTok{dividir_datos}\OperatorTok{==}\OtherTok{FALSE}\NormalTok{)}
\KeywordTok{head}\NormalTok{(autos_testing, }\DecValTok{400}\NormalTok{)}
\end{Highlighting}
\end{Shaded}

\begin{verbatim}
## # A tibble: 123 x 10
##      mpg   cyl  disp    hp weight   acc model.year origin manufacturer
##    <dbl> <dbl> <dbl> <dbl>  <dbl> <dbl>      <dbl>  <dbl> <chr>       
##  1    15     8   350   165   3693    12         70      1 buick       
##  2    14     8   455   225   4425    10         70      1 pontiac     
##  3    26     4   121   113   2234    13         70      2 bmw         
##  4    10     8   360   215   4615    14         70      1 ford        
##  5    16     6   225   105   3439    16         71      1 plymouth    
##  6    19     6   250    88   3302    16         71      1 ford        
##  7    14     8   351   153   4154    14         71      1 ford        
##  8    14     8   318   150   4096    13         71      1 plymouth    
##  9    12     8   383   180   4955    12         71      1 dodge       
## 10    13     8   400   175   5140    12         71      1 pontiac     
## # ... with 113 more rows, and 1 more variable: dividir_datos <lgl>
\end{verbatim}

\begin{enumerate}
\def\labelenumi{\arabic{enumi}.}
\setcounter{enumi}{3}
\tightlist
\item
  Cree un modelo de regresion lineal utilizando el atributo mpg como la
  variable objetivo y en base a las correlaciones observadas en el
  gráfico del punto 2 escoja al menos dos atributos para usarlos como
  variables predictoras para el modelo.
\end{enumerate}

Pista:
\url{https://www.rdocumentation.org/packages/lessR/versions/1.9.8/topics/reg}

Nota: Al crear el modelo utilice el conjunto de datos de entrenamiento
definido en el punto 3.

\begin{Shaded}
\begin{Highlighting}[]
\NormalTok{regresion_data_training =}\StringTok{ }\NormalTok{lessR}\OperatorTok{::}\KeywordTok{reg}\NormalTok{(mpg }\OperatorTok{~}\StringTok{ }\NormalTok{hp }\OperatorTok{+}\StringTok{ }\NormalTok{weight, }
           \DataTypeTok{data=}\NormalTok{autos_training, }\DataTypeTok{dframe=}\NormalTok{autos_training, }
           \DataTypeTok{sig.digits=}\DecValTok{4}\NormalTok{, }\DataTypeTok{res.rows=}\OtherTok{NULL}\NormalTok{, }\DataTypeTok{results=}\StringTok{"brief"}\NormalTok{, }\DataTypeTok{scatter.cor=}\OtherTok{TRUE}\NormalTok{)}
\end{Highlighting}
\end{Shaded}

\includegraphics{Regresion_files/figure-latex/unnamed-chunk-9-1.pdf}
\includegraphics{Regresion_files/figure-latex/unnamed-chunk-9-2.pdf}
\includegraphics{Regresion_files/figure-latex/unnamed-chunk-9-3.pdf}

\begin{Shaded}
\begin{Highlighting}[]
\NormalTok{regresion_data_training}
\end{Highlighting}
\end{Shaded}

\begin{verbatim}
## >>> Suggestion
## # Create an R markdown file for interpretative output with  Rmd = "file_name"
## lessR::reg(mpg ~ hp + weight, data=autos_training, dframe=autos_training, sig.digits=4, res.rows=NULL, results="brief", scatter.cor=TRUE, Rmd="eg")  
## 
## 
##   BACKGROUND
## 
## Data Frame:  autos_training 
##  
## Response Variable: mpg 
## Predictor Variable 1: hp 
## Predictor Variable 2: weight 
##  
## Number of cases (rows) of data:  275 
## Number of cases retained for analysis:  275 
## 
## 
##   BASIC ANALYSIS
## 
##              Estimate    Std Err  t-value  p-value   Lower 95%   Upper 95% 
## (Intercept)    46.316      0.963   48.105    0.000      44.421      48.212 
##          hp    -0.030      0.013   -2.304    0.022      -0.055      -0.004 
##      weight    -0.007      0.001  -11.362    0.000      -0.008      -0.006 
## 
## 
## Standard deviation of residuals:  4.301 for 272 degrees of freedom 
##  
## R-squared:  0.700    Adjusted R-squared:  0.698    PRESS R-squared:  0.693 
## 
## Null hypothesis that all population slope coefficients are 0:
##   F-statistic: 317.577     df: 2 and 272     p-value:  0.000 
## 
## 
##             df    Sum Sq   Mean Sq   F-value   p-value 
##        hp    1  9360.181  9360.181   506.068     0.000 
##    weight    1  2387.571  2387.571   129.086     0.000 
## Model        2 11747.753  5873.876   317.577     0.000 
##  
## Residuals  272  5030.887    18.496 
##  
## mpg        274 16778.640    61.236 
## 
## 
##   RELATIONS AMONG THE VARIABLES
## 
##            mpg    hp weight 
##      mpg  1.00 -0.75  -0.83 
##       hp -0.75  1.00   0.85 
##   weight -0.83  0.85   1.00 
## 
## 
##          Tolerance       VIF 
##       hp     0.281     3.553 
##   weight     0.281     3.553 
## 
## 
##   hp weight    R2adj    X's 
##    1      1    0.698      2 
##    0      1    0.693      1 
##    1      0    0.556      1 
##  
## [based on Thomas Lumley's leaps function from the leaps package] 
##  
## 
## 
##   RESIDUALS AND INFLUENCE
## 
## Data, Fitted, Residual, Studentized Residual, Dffits, Cook's Distance 
##    [sorted by Cook's Distance] 
##    [res_rows = 20, out of 275 rows of data, or do res_rows="all"] 
## --------------------------------------------------------------- 
##            hp   weight    mpg fitted  resid rstdnt dffits cooks 
##    12 225.000 3086.000 14.000 19.085 -5.085 -1.264 -0.475 0.075 
##   275 150.000 2320.000 35.000 26.413  8.587  2.060  0.472 0.073 
##   226  65.000 2110.000 47.000 30.337 16.663  3.995  0.355 0.040 
##   272  52.000 2130.000 44.000 30.590 13.410  3.188  0.337 0.037 
##    85 230.000 4278.000 16.000 11.000  5.000  1.202  0.311 0.032 
##   229  67.000 1850.000 45.000 32.008 12.992  3.084  0.314 0.032 
##   266  85.000 3015.000 38.000 23.717 14.283  3.398  0.304 0.030 
##   174  48.000 1985.000 43.000 31.674 11.326  2.679  0.294 0.028 
##    70 225.000 4951.000 12.000  6.667  5.333  1.268  0.264 0.023 
##   111  72.000 3158.000 15.000 23.151 -8.151 -1.922 -0.264 0.023 
##   110  72.000 3432.000 15.000 21.327 -6.327 -1.496 -0.254 0.021 
##   231 132.000 2910.000 33.000 23.019  9.981  2.353  0.252 0.021 
##   251 105.000 3725.000 27.000 18.395  8.605  2.027  0.245 0.020 
##   218  76.000 2144.000 42.000 29.783 12.217  2.889  0.245 0.020 
##   249  76.000 3160.000 31.000 23.019  7.981  1.879  0.238 0.019 
##   228  65.000 2110.000 41.000 30.337 10.663  2.513  0.223 0.016 
##    88 180.000 3664.000 11.000 16.573 -5.573 -1.316 -0.220 0.016 
##    66 198.000 4952.000 12.000  7.463  4.537  1.070  0.178 0.011 
##   211  77.000 3530.000 25.000 20.526  4.474  1.055  0.177 0.010 
##    18  46.000 1835.000 26.000 32.732 -6.732 -1.579 -0.177 0.010 
## 
## 
##   FORECASTING ERROR
## 
## Data, Predicted, Standard Error of Forecast, 95% Prediction Intervals 
##    [sorted by lower bound of prediction interval] 
##    [to see all intervals do pred_rows="all"] 
## --------------------------------------------------------------- 
##            hp   weight    mpg   pred    sf pi:lwr pi:upr  width 
##   206 125.000 3605.000 19.000 18.600 4.314 10.108 27.092 16.985 
##    12 225.000 3086.000 14.000 19.085 4.559 10.110 28.059 17.949 
##    97 110.000 3632.000 16.000 18.866 4.322 10.358 27.374 17.017 
## ... 
##   129  88.000 2957.000 23.000 24.014 4.314 15.521 32.506 16.985 
##    90 100.000 2901.000 19.000 24.030 4.309 15.548 32.513 16.965 
##   179  85.000 2965.000 20.000 24.050 4.316 15.552 32.547 16.995 
## ... 
##   237  58.000 1755.000 39.000 32.908 4.325 24.394 41.422 17.028 
##    42  69.000 1613.000 35.000 33.527 4.334 24.995 42.059 17.064 
##   104  52.000 1649.000 31.000 33.792 4.328 25.272 42.312 17.040 
## 
## 
## ---------------------------------- 
## Plot 1: Distribution of Residuals 
## Plot 2: Residuals vs Fitted Values 
## Plot 3: ScatterPlot Matrix 
## ----------------------------------
\end{verbatim}

\begin{enumerate}
\def\labelenumi{\arabic{enumi}.}
\setcounter{enumi}{4}
\tightlist
\item
  Realice predicciones utilizando el conjunto de pruebas y evalue el
  resultado con la métrica MSE.
\end{enumerate}

Pista:
\url{https://www.rdocumentation.org/packages/mltools/versions/0.3.5/topics/mse}

\begin{Shaded}
\begin{Highlighting}[]
\CommentTok{# mse(preds = NULL, actuals = NULL, weights = 1, na.rm = FALSE)}
\end{Highlighting}
\end{Shaded}

\begin{enumerate}
\def\labelenumi{\arabic{enumi}.}
\setcounter{enumi}{5}
\tightlist
\item
  Opcional
\end{enumerate}

6.a Pruebe varios modelos que utilicen diferentes variables y comparar
los resultados obtenidos

6.b Investigar como implementar en R las técnicas de preprocesado y
normalización vistas en clase y aplicarlas a los datos antes de pasarlos
al modelo.


\end{document}
